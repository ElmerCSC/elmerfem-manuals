\chapter{Operator splitting in the evolutionary heat equation}

\modinfo{Directory}{TemperatureOperatorSplitting}
\modinfo{Solvers}{HeatSolve, TransportEquationSolver, 
RateOfChangeSolver} 
\modinfo{Tools}{Editor} 
\modinfo{Dimensions}{2D, Transient simulation}


\subsection*{Introduction}

The drawback of the stabilized finite element formulations available
in Elmer to solve the convection-diffusion equation and Navier-Stokes
equations is that these methods are computationally expensive, in particular
when the residual-free-bubbles formulation is used.
In evolutionary problems the reduction of computational cost may be attempted
by applying operator splitting techniques in which the original equation at
each time step is splitted up into subproblems that are
easier to solve numerically.

The aim of this chapter is to provide an illustrative example on
using operator splitting capability in the solution of the 
time-dependent heat equation. 
% and Navier-Stokes equations.   
For the theoretical background of the operator splitting scheme applied here
the reader is referred to Elmer Models Manual and references therein.


%\section{Example I: The evolutionary heat equation}

\subsection*{Case description}

The problem considered here is the solution of the time-dependent 
heat equation in the homogeneous L-shaped plate the geometry of which is shown 
in Figure~\ref{fg:struct1}. 
The density of the material is taken to be unity, while the heat 
conductivity $k$ of the material is taken to be a 
parameter with values ranging between 0.05 and 0.01. The plate is
heated by a constant internal heat source magnitude of which equals to unity. 
The convection velocity field is assumed to be constant with the two 
Cartesian components of the velocity vector equaling to unity. 
The boundaries $\Gamma_i$,
$i=1,4,5,6$, are kept at constant temperature $0$, while the boundaries
$\Gamma_2$ and $\Gamma_3$ are assumed to be insulated, i.e.\ the heat flux
on this part of the boundary vanishes. 

The problem is to solve the temperature distribution in the plate. 
The time interval for the simulation is taken to be [0,2] and the temperature 
of the plate is 0 at the time $t=0$. 


\subsection*{Solution procedure}

Using operator splitting the solution of the heat equation may be replaced 
at each time step by the solution of three subproblems consisting of two 
time-dependent Poisson equations and the convective transport problem.
The effects of diffusion and convection are decoupled by this 
splitting so that 
the diffusion and convection phenomena are taken into account by the steps 
involving the solution of the Poisson equation and the convective transport 
problem, respectively. 

The time-dependent Poisson equations can be solved using the basic heat 
equation solver in Elmer. To avoid the use of stabilized finite element 
formulations in the solution of the convective transport problem this 
subproblem is solved by discretizating an equivalent wave-like equation 
formulation (second order in time equation). 
  
When the operator splitting method is applied,
specific care is needed in prescribing boundary and initial conditions
for the simulation. While the boundary conditions (and initial conditions) 
for the steps involving the solution of the Poisson equation may be defined in 
the usual manner, the boundary conditions for the convective 
transport problem may be prescribed
only on the part of inflow boundary on which the temperature is prescribed. 
In the example the inflow boundary is the union of the boundary segments 
$\Gamma_i$, $i=1,2,5,6$,
so the boundary conditions for the convective transport equation 
should be given on $\Gamma_i$, $i=1,5,6$.
In addition to prescribing the boundary conditions, the rate of change of
the field subject to the convection operator (here temperature) is needed as 
an initial condition at the beginning of pure convection step. This field 
can be solved using a specific solver ({\tt RateOfChangeSolver}) prior to
the solution of the convective transport problem. 
The boundary conditions for this 
solver should be prescribed on the same part of the boundary on which the 
boundary conditions for the convective transport problem are prescribed.    

It should be noted that in connection with the operator splitting method
the user should specify an even number of time steps.
Although the details of implementation need not be known in order to use
the operator splitting ability, it is noted that the the running of two 
successive time steps actually constitutes a single operator splitting scheme
step as described in Elmer Models Manual. 
There are thus three equations (referred to as {\tt Heat Equation},
{\tt Rate Of Change Equation} and {\tt Transport Equation}) which may be 
solved at one time step. 
At odd time steps all these equations are solved meaning that
both the diffusion and convection steps are taken, whereas at even time
steps the solution of the convective transport problem is omitted so that
the diffusion step is performed only. Physically meaningful results
satisfying all the essential boundary conditions may thus be written to
the results file only after even time steps.
 
The mesh for this example was created using Femlab and tools for 
converting meshes from Femlab to Elmer format. The mesh in Elmer format is 
given in the tutorial files. This unstructured mesh
consists of 8458 elements each of which having three nodes.    

The Header section of the solver input file is used to declare the directory
containing the Elmer mesh files:

\ttbegin
Header
  Mesh DB "." "femlab-mesh"
End
\ttend

The Simulation section of the solver input file is used to declare the 
coordinate system and simulation type as well as simulation parameters 
related to the time discretization scheme: 

\ttbegin
Simulation
  Coordinate System = String "Cartesian 2D"
  Simulation Type = String "Transient"
  Timestepping Method = String "Crank-Nicolson"
  Timestep Intervals(1) = Integer 200
  Timestep Sizes(1) = Real 0.01
  Output Intervals(1) = Integer 1
  Steady State Max Iterations = Integer 1
  Post File = File "os-example.ep"
End
\ttend

Here the keyword {\tt Timestepping Method} is used to 
define the time discretization scheme that is used in the solution of the 
time-dependent Poisson equations. Note that one should not use 
multi-step methods in connection with the operator splitting method. 
The time discretization scheme that is used
in the solution of the convective transport problem is fixed and need not
be specified. It should be noted also that
the number of time steps should be even. Since the equations
solved at one time step are not coupled and can thus be solved in sequential 
manner, {\tt Steady State Max Iterations} may be taken to be 1.     

The Body section of the solver input file is used to declare integer 
identifiers for body forces, equations, initial conditions and materials: 

\ttbegin
Body 1
  Body Force = Integer 1
  Equation = Integer 1
  Initial Condition = Integer 1
  Material = Integer 1
End
\ttend

The Equation section of the solver input file in turn has the following 
declaration

\ttbegin
Equation 1
  Active Solvers(3) = Integer 1 2 3
End
\ttend
indicating that three equations are to be solved using solvers with 
the integer identifiers 1, 2 and 3.   
Accordingly, three Solver sections are needed.  
The first Solver section is used for the Poisson equation and has the 
following declarations:

\ttbegin
Solver 1
  Equation = String "Heat Equation"
  Variable = String "Temperature"
  Variable Dofs = Integer 1
  Linear System Solver = String "Iterative"
  Linear System Iterative Method = String "BiCGStab"
  Linear System Max Iterations = Integer 350
  Linear System Convergence Tolerance = Real 1.0e-08
  Linear System Abort Not Converged = Logical True
  Linear System Preconditioning = String "ILU0"
  Steady State Convergence Tolerance = Real 1.0e-05
  Stabilize = Logical False
  Bubbles = Logical False
End
\ttend
Note that there is no need to use stabilization, so the values of the keywords 
{\tt Stabilize} and {\tt Bubbles} may be set to be {\tt False} to reduce the
computational cost. 

The remaining two Solver sections are related to the convective transport
problem. The first one of these sections is used to declare the solver
parameters related to the problem ({\tt Rate Of Change Equation}) the 
solution of which gives the approximation to the rate of change of the 
temperature at the beginning of pure convection step. The contents
of this solver section is 

\ttbegin
Solver 2
  Equation = String "Rate Of Change Equation"
  Procedure = File "RateOfChange" "RateOfChangeSolver"
  Variable = String "Udot0"
  Variable Dofs = Integer 1   
  Advection = String "Constant"
  Advection Variable = String "Temperature" 
  Linear System Solver = String "Iterative"
  Linear System Iterative Method = String "BiCGStab"
  Linear System Max Iterations = Integer 350
  Linear System Convergence Tolerance = Real 1.0e-08
  Linear System Abort Not Converged = Logical True
  Linear System Preconditioning = String "ILU0"
  Steady State Convergence Tolerance = Real 1.0e-05
End
\ttend

Here the keyword {\tt Advection Variable} is used to declare the quantity
which is subject to the convection operator, while the keyword {\tt Advection}
is used to define the type of the velocity vector.
The name {\tt Udot0} is used for the solution of this problem.

Finally, the Solver section for the wave-like equation, which is equivalent to 
the convective transport equation, has the following declarations:

\ttbegin
Solver 3  
  Equation = String "Transport Equation"
  Procedure = File "TransportEquation" "TransportEquationSolver"
  Time Derivative Order = Integer 2
  Variable = String "U"
  Variable Dofs = Integer 1  
  Advection = String "Constant"
  Advection Variable = String "Temperature" 
  Rate Of Change Equation Variable = String "Udot0" 
  Linear System Solver = String "Iterative"
  Linear System Iterative Method = String "BiCGStab"
  Linear System Max Iterations = Integer 350
  Linear System Convergence Tolerance = Real 1.0e-8
  Linear System Abort Not Converged = Logical True
  Linear System Preconditioning = String "ILU0"
  Steady State Convergence Tolerance = Real 1.0e-05
End
\ttend

The name {\tt U} is used for the solution of the convective transport problem. 
The value of the keyword {\tt Time Derivative Order} must be 2.
The use of the keywords {\tt Advection Variable} and {\tt Advection}
is similar to that explained in connection with the second solver section.  

The Material section is used to declare the material properties and,
as the velocity vector in the convection operator is of constant type,
also the components of the velocity vector. In the case $k=0.01$ the contents 
of this section is  

\ttbegin
Material 1
  Density = Real 1
  Heat Capacity = Real 1
  Heat Conductivity = Real 0.01
  Advection Velocity 1 = Real 1
  Advection Velocity 2 = Real 1
End
\ttend

Body Force section is used to declare the body force in the Poisson
equations.

\ttbegin
Body Force 1
  Heat Source = Real 1
End
\ttend

Finally, the initial conditions and boundary conditions are specified. 
The temperature at $t=0$ is defined by giving the declaration  

\ttbegin
Initial Condition 1
  Temperature = Real 0
End
\ttend

Two Boundary Condition sections are needed. The first one is used 
to prescribe the boundary conditions on the part of inflow boundary where
the temperature is given: 

\ttbegin
Boundary Condition 1
  Target Boundaries(3) = Integer 1 2 4 
  Temperature = Real 0
  Udot0 1 = Real 0
  U 1 = Real 0
End
\ttend

The rate of change of the temperature is zero on this part of the boundary 
as the temperature is kept fixed. Thus the zero boundary 
conditions for {\tt Udot0} are defined. The boundary value of the 
solution to the convective transport problem should equal to the
temperature. Therefore zero boundary conditions for {\tt U} are also defined.
  
The second Boundary Condition section is used to define the Dirichlet boundary 
conditions on the outflow boundary: 

\ttbegin
Boundary Condition 2
  Target Boundaries(1) = Integer 3
  Temperature = Real 0
End
\ttend

Note that the zero heat flux condition
need not be specified explicitly. Similarly, the treatment of 
the outflow boundary conditions for the wave-like
equation are handled automatically by the code and need not be specified. 


\subsection*{Results}

From a numerical point of view the solution of the problem becomes 
increasingly difficult as the heat conductivity $k$ tends to zero. 
In order to examine possible dependence on the heat conductivity parameter
the problem was solved in three cases with $k$ taking values 0.05, 0.025 
and 0.01. For comparison the same case was solved using three alternative 
methods. Here the operator splitting scheme is referred to as OS, S is the 
stabilized finite element method and RFB is the method based on the 
residual-free-bubbles formulation. In the case of methods S and RFB the 
simulation was performed using 100 time steps which corresponds to the number 
of 200 time steps used in the case of operator splitting scheme.
 
The maximum temperature at $t=2.0$ is recorded in 
Table~\ref{methodvstemperature}. The CPU time used in the simulation to
obtain solution for a certain value of the heat conductivity is
shown in Table~\ref{methodvscpu}.
   
\begin{table}  
\caption{The maximum temperature at $t=2.0$. For comparison 
the maximum temperature according to the steady state solution is also 
recorded.}
\label{methodvstemperature}
\begin{center}
\begin{tabular}{lccc}\hline
Method            &        &  $k$   &                \\ \hline
                  & 0.05   & 0.025  & 0.01           \\ \hline
OS                & 1.0235 & 1.0424 & 1.0677         \\
S                 & 1.0269 & 1.0436 & 1.0418         \\
S(steady state)   & 1.0279 & 1.0437 & 1.0418         \\
RFB               & 1.0271 & 1.0458 & 1.0853         \\
RFB(steady state) & 1.0286 & 1.0462 & 1.1050         \\ \hline
\end{tabular}
\end{center}
\end{table}

\begin{table} 
\caption{CPU time used by the different methods for simulation ($k=0.01$).}
\label{methodvscpu}
\begin{center}
\begin{tabular}{lc}\hline
Method            & CPU     \\ \hline
OS                & 211.13  \\
S                 & 82.16   \\
RFB               & 340.70  \\ \hline
\end{tabular}
\end{center}
\end{table}


\subsection*{Discussion}

The benefit of the wave-like equation formulation of the convective 
transport problem is that this formulation can be discretizated without
using stabilized finite element formulations. Thus all the subproblems
arising from operator splitting can be solved using standard FE
techniques. On the other hand,
the expense of this approach is that one is lead to handle the second
order in time equation. 

In the current implementation of the method the 
wave-like equation is discretizated in time
using the Crank-Nicolson scheme which is expected to perform well if 
the solution to the convective transport problem is smooth. Spurious
oscillations may however occur in the case of a rough solution changing 
rapidly in short length scales. The user of the method should be
aware that the deterioration of accuracy may thus occur if the solution is
not smooth and $\epsilon=k/||\vec u ||_\infty \longrightarrow 0$,
meaning that convection dominates.  
   
In the cases considered convection dominates, the parameter $\epsilon$ 
ranging between $3.5\cdot 10^{-2}$ and $7.1\cdot 10^{-3}$,
and the solution has also sharp boundary 
layer near the upper edge. No spurious oscillations are however detected in
the solution. Nevertheless, the results recorded in 
Table~\ref{methodvstemperature} indicate that the differences between the
methods become larger as $\epsilon \longrightarrow 0$. 
 
In view of the results shown in Table~\ref{methodvscpu} the accuracy of the 
operator splitting scheme in predicting the maximum temperature is comparable 
to that of the residual-free-bubbles method while the computational 
cost measured in CPU time is reduced by approximately 40~\% when 
the operator splitting scheme is used. 

It should be noted that the operator splitting scheme has 
the favourable feature that small spurious oscillations present in the 
solution after pure convection step may naturally be damped by the step 
involving diffusion phenomena. Note also that each of the subproblems arising 
from the operator splitting may be
solved very efficiently using multigrid techniques, whereas robust multigrid 
solvers amenable to solving linear systems arising from direct discretization 
of the convection-diffusion equation are still to be found. This makes 
the operator splitting scheme attractive in the solution of problems 
having a very large number of unknowns.   




%\section{Example II: The evolutionary Navier-Stokes equations}



