\chapter*{Instructions for the Elmer GUI tutorials}

Here are some instructions for following the GUI tutorials:
\begin{itemize}
\item All the needed input files should be available among the \texttt{ElmerGUI/samples} 
  directory that should have come with the installation. Look under a subdirectory
  named after the suffix of the sample file.
%
\item The instructions written in \texttt{verbatim} refer to operations 
  with the GUI. Intendation means step in the menu hierarchy. 
  The instructions should not be mixed with those in the command file. 
%
\item The menu structure for the default set of equations is located 
  in directory \texttt{edf}, there are a few additional ones in 
  directory \texttt{edf-extra}. These may be copied to the directory \texttt{edf}
  permanently, or be appended to the menus while running the ElmerGUI. 
%
\item The default menu structure may differ from the configuration used
  when writing the tutorial. Hence the user is encouraged to check by herself 
  whether the menu structures exist or not. 
%
\item After having once defined the case you may go to the working directory 
  and launch ElmerSolver from command-line. There you may edit the .sif file 
  to alter the parameters.
%
\item Manual alteration to the sif file will not be communicated to the
  ElmerGUI project. All editions will be overrun by the GUI when saving the 
  project.
%
\item It is assumed that the default method for visualization is Paraview (or ViSit) using \texttt{.vtu} format.
  \texttt{ElmerPost} is with \texttt{.ep} format is no longer available in most installations.
  Unfortunately, all of the tutorials do not yet use Paraview for visualization.
  
\item Since ElmerGUI v. 9.0 the VTK widget that comes with
  ElmerGUIhas been upgraded to use \texttt{.vtu} files. It may still be an ideal visualization tool
  for most straight-forward demonstrations.
%
\item The cases have been run a number of times but errors are still possible.
  Reporting them to elmeradm@csc.fi, for example, is greatly appreciated. 
  
\end{itemize}







