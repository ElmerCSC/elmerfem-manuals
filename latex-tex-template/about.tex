\chapter*{Template For ElmerGUI Tutorials}

\section*{About this document}

This document, Template For ElmerGUI Tutorials, is intended to provide an easy to follow template to create new ElmerGUI tutorials.  The present manual corresponds to Elmer software version~\elmerversion{}.  Latest documentations and program versions of Elmer are available (or links are provided) at \url{http://www.csc.fi/elmer}. 

\section*{Copyright information}

This document is licensed under the Creative Commons Attribution-NonCommerical 3.0 License.  To view a copy of this license, visit \url{http://creativecommons.org/licenses/by-nc/3.0/}.\\

Initially this Template For ElmerGUI Tutorials has been written by Rich Bayless.  External contributions to the Template are welcome.

\section*{Usage tips}

This Template For Tutorials is designed to help make it easy to create a new tutorial using ElmerGUI.  Having an outline of a tutorial should allow for simpler development of new tutorials, and will help keep the look and feel of new tutorials in unison with existing tutorials.

Anywhere you see `zzz', it means that section needs to be replaced.  Any text that begins with `zzz', should be inspected and replaced by the appropriate text for your tutorial.  Think of `zzz' as a place holder where you should `fill in the blanks'.

After creating and editing a new tutorial tex document, be sure to use 'find' to check that all of the 'zzz' place holders have been found and removed.

The document, TexStart.pdf, is a companion document and includes specific instructions on using Latex to edit Elmer documenation.  Be sure to review and use the content of TexStart.pdf.\\

All of the available ElmerGUI Equation Definition Files (EDFs) are installed during the installation of Elmer.

\noindent The EDFs are, by default, located in Linux in:

\texttt{\$ELMER\_HOME/share/ElmerGUI/edf}

\texttt{\$ELMER\_HOME/share/ElmerGUI/edf-extra}

\noindent and in Windows, located in:

\texttt{C:/Program Files/Elmer 9.0-Release/share/ElmerGUI/edf}

\texttt{C:/Program Files/Elmer 9.0-Release/share/ElmerGUI/edf-extra}\\

The New Project screen will automatically locate the EDFs loaded from the above locations.  Although it is not needed, if desired you can change which EDFs are loaded from the edf file location and which extra EDFs are loaded from the edf-extra file location.  Simply cut and paste any of the edf menu files from one folder to the other.   ElmerGUI will then load the files upon start up, and the New Project screen will reflect your choices.


